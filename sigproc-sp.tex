% THIS IS SIGPROC-SP.TEX - VERSION 3.1
% WORKS WITH V3.2SP OF ACM_PROC_ARTICLE-SP.CLS
% APRIL 2009
%
% It is an example file showing how to use the 'acm_proc_article-sp.cls' V3.2SP
% LaTeX2e document class file for Conference Proceedings submissions.
% ----------------------------------------------------------------------------------------------------------------
% This .tex file (and associated .cls V3.2SP) *DOES NOT* produce:
%       1) The Permission Statement
%       2) The Conference (location) Info information
%       3) The Copyright Line with ACM data
%       4) Page numbering
% ---------------------------------------------------------------------------------------------------------------
% It is an example which *does* use the .bib file (from which the .bbl file
% is produced).
% REMEMBER HOWEVER: After having produced the .bbl file,
% and prior to final submission,
% you need to 'insert'  your .bbl file into your source .tex file so as to provide
% ONE 'self-contained' source file.
%
% Questions regarding SIGS should be sent to
% Adrienne Griscti ---> griscti@acm.org
%
% Questions/suggestions regarding the guidelines, .tex and .cls files, etc. to
% Gerald Murray ---> murray@hq.acm.org
%
% For tracking purposes - this is V3.1SP - APRIL 2009


% require the `acm_proc_article-sp.cls` file.
\documentclass{acm_proc_article-sp}

\begin{document}

\title{Matain-e-nator: \ttlit{Make the World a Better Place}}
%
% You need the command \numberofauthors to handle the 'placement
% and alignment' of the authors beneath the title.
%
% For aesthetic reasons, we recommend 'three authors at a time'
% i.e. three 'name/affiliation blocks' be placed beneath the title.
%
% NOTE: You are NOT restricted in how many 'rows' of
% "name/affiliations" may appear. We just ask that you restrict
% the number of 'columns' to three.
%
% Because of the available 'opening page real-estate'
% we ask you to refrain from putting more than six authors
% (two rows with three columns) beneath the article title.
% More than six makes the first-page appear very cluttered indeed.
%
% Use the \alignauthor commands to handle the names
% and affiliations for an 'aesthetic maximum' of six authors.
% Add names, affiliations, addresses for
% the seventh etc. author(s) as the argument for the
% \additionalauthors command.
% These 'additional authors' will be output/set for you
% without further effort on your part as the last section in
% the body of your article BEFORE References or any Appendices.

\numberofauthors{3} %  in this sample file, there are a *total*
% of EIGHT authors. SIX appear on the 'first-page' (for formatting
% reasons) and the remaining two appear in the \additionalauthors section.
%
\author{
% You can go ahead and credit any number of authors here,
% e.g. one 'row of three' or two rows (consisting of one row of three
% and a second row of one, two or three).
%
% The command \alignauthor (no curly braces needed) should
% precede each author name, affiliation/snail-mail address and
% e-mail address. Additionally, tag each line of
% affiliation/address with \affaddr, and tag the
% e-mail address with \email.
%
% 1st. author
\alignauthor
Xin Liu\\
       \affaddr{University at Buffalo}\\
       \affaddr{1932 Wallamaloo Lane}\\
       \email{xliu36@buffalo.edu}
% 2nd. author
\alignauthor
John Longanecker\\
       \affaddr{University at Buffalo}\\
       \affaddr{P.O. Box 1212}\\
       \email{webmaster@marysville-ohio.com}
% 3rd. author
\alignauthor
Juehui Zhang\\
       \affaddr{University at Buffalo}\\
       \affaddr{Hekla, Iceland}\\
       \email{larst@affiliation.org}
%\and  % use '\and' if you need 'another row' of author names
}

\maketitle

\begin{abstract}
Our goal is to improve the overall quality of a facility as well as decrease an organization�s overall operating expenses. 
By letting those who maintain facilities know about problems sooner. They can react quicker and more efficiently if they have better information about the status of their facilities. 
\textit{Maintain-e-nator} provides a cell phone application to report problems as well as a web interface to 
allow maintenance workers to be notified of new problems.
\end{abstract}

% A category with the (minimum) three required fields
\category{H.4}{Information Systems Applications}{Miscellaneous}
%A category including the fourth, optional field follows...
\category{D.2.8}{Software Engineering}{Metrics}[complexity measures, performance measures]

\terms{Applications}

 % NOT required for Proceedings
\keywords{ACM proceedings, \LaTeX, text tagging}

%==========
% INTRODUCTION
%==========
\section{Introduction}
Eventually everything breaks. Nothing lasts forever. Buildings start as brand new but eventually break down. 
Roads start as smooth, but eventually develop potholes. 
These breakdowns can sometimes be ignored like a squeaky door, but others can cause safety and health risks. 
If the stairs of a building are in disrepair they could cause a tripping hazard for other people.

%==========
% Motivation
%==========
\section{Motivations}
It is easy for a large organization to be unaware of all the maintenance problems that their facilities have. 
Some obscure room may need a light bulb replaced but those that work in that area do not report the problem or are not around when the build is exhibiting its behaviors. 
So the people who take a night class are the only ones aware of the problem. 
They do not know where to submit a problem and are often not willing to do the necessary research to find out how to report a problem.

These problems are not limited to the indoors. They can also involve roads, landscaping, sidewalks, outdoor sports facilities.

\subsection{What is a problem?}
We consider a maintenance problem anything that can hurt someone as well as something that detracts from the overall quality of the facilities. 
So a dirty floor or table could be considered a problem. 
At the end of the day the users who submit a problem are the ones who are deciding what a problem is. 
Who better to determine a problem then those who actually use the facilities?

\subsection{Goal}
The overall goal is to make maintenance workers aware of the problems that exists on their property. 
Are android application and web interface will not promise cleaner facilities. Our goal is to help those that manage a property.

%==========
% DESIGN
%==========
\section{Design}
We use \cite{django} as our web server framework.

%==========
% CONCLUSION
%==========
\section{Conclusions}
This paragraph will end the body of this sample document.
Remember that you might still have Acknowledgments or
Appendices; brief samples of these
follow.  There is still the Bibliography to deal with; and
we will make a disclaimer about that here: with the exception
of the reference to the \LaTeX\ book, the citations in
this paper are to articles which have nothing to
do with the present subject and are used as
examples only.

\section{Acknowledgments}
This section is optional; it is a location for you
to acknowledge grants, funding, editing assistance and
what have you.  In the present case, for example, the
authors would like to thank Gerald Murray of ACM for
his help in codifying this \textit{Author's Guide}
and the \textbf{.cls} and \textbf{.tex} files that it describes.



\bibliographystyle{abbrv}
\bibliography{sigproc}
\balancecolumns
% That's all folks!
\end{document}
